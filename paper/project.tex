%\documentclass{amsart}

\documentclass{article}
\usepackage[letterpaper,hmargin=15mm,vmargin=20mm]{geometry}
\usepackage[nosetup, colorlinks]{tony}
\usepackage{graphicx}

\usepackage{amsmath,amssymb}
\usepackage{siunitx}

\usepackage{mathpazo}
\usepackage{microtype}
\usepackage{multicol}

\usepackage{diagbox}

\usepackage{color}
\usepackage[dvipsnames]{xcolor}
%\usepackage[printwatermark]{xwatermark}
%\newwatermark*[allpages,color=gray!50,angle=45,scale=3,xpos=0,ypos=0]{DRAFT}

\usepackage{tikz}
\usetikzlibrary{arrows}

\DeclareMathOperator{\sgn}{sgn}
\DeclareMathOperator{\NLL}{NLL}
\newcommand{\sind}[1]{^{(#1)}}

\title{Classifying airline departure delays with ensemble methods}
\author{Tony Zhang and Menghua Wu}
\date{December 16, 2016}

\begin{document}
\maketitle

\begin{multicols}{2}

% % % % % % % % % %
%    INTRODUCTION
% % % % % % % % % %

\section{Background}
\label{sec:bg}

Often, people purchase plane tickets
to minimize monetary cost.
However, for many circumstances,
there is also utility in
in minimizing the risk that a flight will be late.
Thus, the underlying motivation of this paper
is to provide a metric of expected flight delay
as a means of comparing different flights.

Each year,
the United States Bureau of Transportation Statistics (BTS)
compiles comprehensive datasets
regarding the nation's transportation systems,
including aviation, maritime, highway, and rail.
In this paper, we focus on the airline dataset,
which reports on a wide range of variables including
airline, air carrier, origin, and destination, and flight delays.
We downloaded exhaustive airline data
from June 2015 to May 2016, inclusive.
Each month's data provides
approximately 500,000 samples of individual flight data,
some of which may be incomplete.
We used these data to build a classifier
that predicts whether or not a flight's departure
is expected to be delayed.
We combined several methods of classification
and compare their results here.

\section{Random forest classification}

We leveraged existing random forest implementations
and determined the optimal parameters for classifying our data.

There may be strong relationships between
month of year and airline delay,
as the holiday season is often associated with
inclement weather and an increase in delays.
Therefore, we trained a separate random forest classifier
for each month, in additional to a single classifier
for the combined data.
For each month,
the resultant model was an average of
that month's classifier and the combined classifier,
weighted by validation error.

% TODO INSERT PLOT and data

On the testing data,
the combined classifier performed better
than each classifier alone,
perhaps due to increased generalization.

\subsection{Tree depth}

We experimented with increasing the xxx

\subsection{Masking features}

xxx maximum number of features to consider when splitting

\subsection{Early stopping}

or the impurity setting

\subsection{Whatever you call ``warm start''}

\section{Dataset transformations}
\label{sec:dataset}

The raw dataset from the BTS
includes unconventional features,
so we preprocessed our dataset
to improve classification feasibility.

Use best parameters from previous section.

\subsection{Discrete features this doesn't go here}

Discrete features in our dataset included
air carrier, airline, origin, and destination.
We mapped these to one-hot vectors,
whose dimensions were flexible based on
the number of categories present in the dataset.

\subsection{Cyclic temporal features}

Temporal features, such as time of day,
are naturally cyclic.
Representing them on a linear scale
neglects the fact that

\subsection{Geographic relations within data}

So far, we have considered origin and destination data
as discrete values,
each airport with an equal probability of causing delays.
That is, the distance between any two airports
(as one-hot vectors) is the same.
However, this may not be the case in the real world.

\subsection{Geographic displacement}

East west vs. vice versa, etc.

\section{Comparison with other methods}

We compared our ensemble classifier
with other types of classifiers,
including logistic regression, a neural network,
and a kernelized Pegasos classifier
with the Gaussian RBF kernel.
Due to the large size of our dataset,
we were limited to
stochastic and minibatch methods.

\end{multicols}

\end{document}

