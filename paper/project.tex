%\documentclass{amsart}

\documentclass{article}
\usepackage[letterpaper,hmargin=15mm,vmargin=20mm]{geometry}
\usepackage[nosetup, colorlinks]{tony}
\usepackage{graphicx}

\usepackage{amsmath,amssymb}
\usepackage{siunitx}

\usepackage{mathpazo}
\usepackage{microtype}
\usepackage{multicol}

\usepackage{diagbox}

\usepackage{color}
\usepackage[dvipsnames]{xcolor}
%\usepackage[printwatermark]{xwatermark}
%\newwatermark*[allpages,color=gray!50,angle=45,scale=3,xpos=0,ypos=0]{DRAFT}

\usepackage{tikz}
\usetikzlibrary{arrows}

\DeclareMathOperator{\sgn}{sgn}
\DeclareMathOperator{\NLL}{NLL}
\newcommand{\sind}[1]{^{(#1)}}

\title{Classifying airline departure delays with ensemble methods}
\author{Tony Zhang and Menghua Wu}
\date{December 16, 2016}

\begin{document}
\maketitle

\begin{multicols}{2}

% % % % % % % % % %
%    INTRODUCTION
% % % % % % % % % %

\section{Background}
\label{sec:bg}

Often, people purchase plane tickets
to minimize monetary cost.
However, for many circumstances,
there is also utility in
in minimizing the risk that a flight will be late.
Thus, the underlying motivation of this paper
is to provide a metric of expected flight delay
as a means of comparing different flights.

Each year,
the United States Bureau of Transportation Statistics (BTS)
compiles comprehensive datasets
regarding the nation's transportation systems,
including aviation, maritime, highway, and rail.
In this paper, we focus on the airline dataset,
which reports on a wide range of variables including
airline, air carrier, origin, and destination, and flight delays.
We downloaded exhaustive airline data
from June 2015 to May 2016, inclusive.
Each month's data provides
approximately 500,000 samples of individual flight data,
some of which may be incomplete.
We used these data to build a classifier
that predicts whether or not a flight's departure
is expected to be delayed.
We combined several methods of classification
and compare their results here.

\section{Dataset transformations}
\label{sec:dataset}

The raw dataset from the BTS
includes unconventional features,
so we preprocessed our dataset
to improve classification feasibility.

\subsection{Discrete features}

Discrete features in our dataset included
air carrier, airline, origin, and destination.
We mapped these to one-hot vectors,
whose dimensions were flexible based on
the number of categories present in the dataset.

\subsection{Continuous linear features}

xxx

\subsection{Cyclic features}

xxxx

\section{Random forest classification}

We leveraged existing random forest implementations
and experimented with different parameters.

\section{Comparison with other methods}

\end{multicols}

\end{document}

